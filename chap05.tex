%%!TEX root = ./marylie.tex
%%--.----1----.----2----.----3----.----4----.----5----.----6----.----7----.-!

\chapter[Format of MaryLie Master Input File \& Use of PREP]{Format of
MaryLie Master Input File and Use of PREP}\index{PREP} \index{master input file}

\section{General Instructions} As indicated in section 4.3.1, file 11 may
be prepared with the aid of the program PREP\@. The use of PREP is not
required, but for the beginner it will ease the labor of initially inputing a large lattice.
Slight modifications of an existing lattice are more easily accomplished by
directly editing the file (usually file 11) in which it is stored. Once the
reader has observed the Master Input File as prepared by PREP, she or he
will be able to infer its format, and if desired, will be able to make up
such a file directly.

%\numberbysubsection
\subsection{PREP Input} PREP runs interactively, and receives input and
communicates with the user through the terminal (files 5 and 6). Also, if
desired, PREP can receive information from file 9. See section 5.3. As
indicated in section 4.3.2, PREP uses free format input.

\subsection{PREP Output} The output of PREP is written on file 10. The
contents of file 10 must subsequently be copied into file 11 to be used for
a \Mary run, or into file 9 to restart a PREP run.

\section{Use of Help Feature} For the convenience of the user, PREP
contains help information that can be accessed at any time through
appropriate help commands.

When PREP is executed, the first thing it replies is
\begin{footnotesize}
\begin{verbatim}
**PREP 3.0** Master Input File preparation for MARYLIE 3.0
Copyright 1987 Alex J. Dragt
All rights reserved
\end{verbatim}
\end{footnotesize}
\pagebreak It then provides opportunity to receive brief instructions about
the use of the help feature:
\begin{footnotesize}
\begin{verbatim}
Do you want instructions? (y,n) <n> y
\end{verbatim}
\end{footnotesize} PREP runs interactively, and help is available at any
time. Help may be obtained in any of three ways depending on the
circumstances:
\begin{itemize}
\item If PREP expects an answer or requests input in the form of a
character string, you may type ? or ?help instead of the expected response.
After the help session, PREP will return to where it was and re-ask the
same question or re-request the same input.

\item If PREP is expecting a number, you may get help by responding with a
number that equals 1.e+30 or more. Again, after the help session, PREP will
return to where it was.

\item At various places in a PREP run, as will be evident, you will be able
to ask for help directly.
\end{itemize} There are several levels in help. To return directly from
help, type `exit'. To backtrack through the various levels, type `back'. To
re-view a current level, type `see'.

The further operation of the help feature is meant to be self explanatory.
The user is urged to try this feature whenever a question arises.


\section{Input From External File} At the user's option, PREP can receive
information from file 9. The structure of file 9 is the same as that of
files 10 and 11. This feature makes it possible for a user to begin a PREP
run with the results of a previous PREP run, or a user prepared file 11.
For an example of the use of file 9, see section 5.14. PREP gives
opportunity to use file 9 by asking the following question at the beginning
of a run:
\begin{footnotesize}
\begin{verbatim}
Read a PREP data set from file 9? (y,n,?) <n>
\end{verbatim}
\end{footnotesize}

\section{Modes, Editors, and Related Commands} After offering the options
of instructions and the use of a data set at the beginning of a run, PREP
goes into {\em control} mode as indicated by the lines below:
\begin{footnotesize}
\begin{verbatim}
 Control mode. Type
cmts,beam,menu,line,lump,loop,labor,summ,file,exit,abort or ?:
\end{verbatim}
\end{footnotesize} PREP runs in several modes. These modes enable the user
to specify information concerning the comment, beam, menu, lines, lumps,
loops, and labor components of file 11. These components have already been
described briefly in section 4.4.1. Listed below are the associated modes,
their purpose, and the sections describing these modes in more detail.

\begin{center}
\begin{tabular}{|c|l|} \hline Mode & Purpose/Section \\ \hline comment
(cmts) & To allow the user to write comments about the \\ & system under
study and the computations to be made.\\ & Section 5.5. \\ & \\ beam & To
input the \#beam component which describes \\ & the magnetic rigidity and
relativistic $\beta$ and \\ & $\gamma$ factors of the beam, and its charge,
and \\ & to specify the scale length to be used. \\ & Section 5.6. \\ & \\
menu & To input the \#menu component which contains \\ & a list of
beam-line elements and commands. \\ & Section 5.7. \\ & \\ line & To input
the \#lines component which contains \\ & a list of names for collections
of elements \\ & and/or commands drawn from the menu. \\ & Section 5.8. \\
& \\ lump & To input the \#lumps component which contains a \\ & list of
items from the menu and/or lines that are \\ & to be combined together to
form individual maps.\\ & Section 5.9. \\ & \\ loop & To input the \#loops
component which contains \\ & information about special tracking or other
\\ & operations. Section 5.10. \\ & \\ labor & To input the \#labor
component which \\ & specifies the system to be studied, and the \\ &
operations and calculations to be performed.\\ & Section 5.11. \\ \hline
\end{tabular}
\end{center}

In addition to the modes listed above, PREP has a {\em control} mode. It is
activated by typing {\em ctrl}. Its purpose is to give access to the other
modes as well as the special commands {\em summ}, {\em file}, {\em exit},
{\em abort}, and {\em ?help}. The {\em ?help} command has already been
described in section 5.2. The purpose of the other commands, and the
sections in which they are described in fuller detail, are listed below.

\begin{center}
\begin{tabular}{|c|l|} \hline Command & Purpose/Section \\ \hline summ & To
obtain a listing (summary) of what has been \\ & accomplished to date in a
PREP session. \\ & Section 5.12. \\ & \\ file & To write on file 10
whatever has been specified \\ & to date about the comments, beam, menu,
lines, \\ & lumps, loops, and labor components. \\ & Section 5.13. \\ & \\
exit & To cause normal termination of a PREP run. \\ & Section 5.14 \\ & \\
abort & To cause abnormal termination of a PREP run. \\ & Section 5.14. \\
\hline
\end{tabular}
\end{center}

The use of beam mode is relatively simple, and any mistakes made while in
this mode are easily rectified by rerunning the mode. See section 5.6. The
use of the comments, menu, lines, lumps, loops, and labor modes is more
complex. Each of these modes has a special editor that enables the user to
see what has been accomplished, make changes, and return to the control
mode. These special editors are described in the same sections as their
respective modes.

\section{Comment (cmts) Mode}\index{comment} The {\em comment} mode allows the user to
write a comments section about the system under study and the calculations
to be performed. The use of the \#comments component of the Master Input
File is optional.

\subsection{Format of \#comment Component} As presently configured, the
\#comment component may contain up to 100 lines. Each line may contain up
to 72 characters.

\subsection{Sample Conversation in Comments Mode} Below is a sample
conversation with PREP while in {\em comment} mode. The user has put in a
one line comment, returned to {\em control} mode by way of the {\em
comment} editor, and issued a {\em summ} command to see the proposed
contents of the Master Input File up to this point.
\begin{footnotesize}
\begin{verbatim}
Type cmts,beam,menu,line,lump,loop,labor,summ,file,exit,abort or ?:
cmts
comments input mode:
input next comment line (type '#' to edit)
This is a demonstration of the use of PREP.
input next comment line (type '#' to edit)
#
Comment edit mode
type p,pc,pl,u,n,d,dl,ud,i,ib,?, or ctrl :
ctrl
Control mode.
Type cmts,beam,menu,line,lump,loop,labor,summ,file,exit,abort
or ?:
summ
#comment
 This is a demonstration of the use of PREP.
#beam
 0.0000000000000000E+00
 0.0000000000000000E+00
 0.0000000000000000E+00
 0.0000000000000000E+00
#menu
Control mode.
\end{verbatim}
\end{footnotesize}

\subsection{Comment Editor} PREP has a simple line editor for correcting
mistakes in the \#comment component. To assist the user, the editor has a
``viewing window'' that sits over what is referred to as the ``current''
line. The editor for \#comment is activated by typing `\#'. The available
commands for the \#comment editor are listed below. Here (CR) denotes
``Carriage Return''.

\begin{table}
\begin{center}
\begin{tabular}{|c|l|l|} \hline
\multicolumn{3}{|c|}{Comments Editor Commands} \\ \hline Command & Usage &
Result \\ \hline p & p(CR) & {\underline p}rint all of \#comment component
with line \\ & & numbers. After this command is executed \\ & & the current
line is the last line. \\ & & \\ pc & pc(CR) & {\underline p}rint
{\underline c}urrent line. \\ & & \\ pl & pl(CR) & {\underline p}rint
{\underline l}ines m through n. After this \\ & m,n(CR) & command is
executed, the current line \\ & & is line n. Thus, this command may be \\ &
& used with m=n to place the viewing \\ & & window over line n. \\ & & \\ u
& u(CR) & Move viewing window {\underline u}p one line. \\ & & \\ n & n(CR)
& Move viewing window to the {\underline n}ext line. \\ & & \\ d & d(CR) &
{\underline d}elete the current line. Viewing \\ & & window now sits over
the next line. \\ & & \\ dl & dl(CR) & {\underline d}elete {\underline
l}ines m through n. Viewing \\ & m,n(CR) & window now sits over the old
line n+1. \\ & & \\ ud & ud(CR) & {\underline u}n{\underline d}elete before
line m. Inputs before\\ & m(CR) & line m the most recently deleted set \\ &
& of lines (by the command {\em d} or {\em dl}).\\ & & Viewing window now
sits over the old line m. \\ & & \\ i & i(CR) & {\underline i}nput after
the current line. PREP \\ & & goes into {\em comments} input mode. To \\ &
& return to the {\em comments} editor, type \#. \\ & & \\ ib & ib(CR) &
{\underline i}nput {\underline b}efore line m. PREP goes into\\ & m(CR) &
{\em comments} input mode. To return to the \\ & & {\em comments} editor,
type \#. \\ & & \\ ? & ?(CR) & Get help from PREP. \\ & & \\ ctrl &
ctrl(CR) & Return to {\em control} mode. \\ \hline
\end{tabular}
\end{center}
\end{table}

To leave the {\em comments} editor and return to {\em control} mode, type
{\em ctrl}. The proposed contents of the Master Input File as prepared by
PREP up to this point may then be examined by entering the command {\em
summ}. Modifications in the \#comment component may be made at any time by
returning to the {\em comments} mode by way of the {\em control} mode.

If extensive changes in the \#comment component are required, it may be
easier to make them by using a local Editor on files 10 or 11 after
completion of a PREP run.

\subsection{Interspersed Comments}\index{interspersed comments} Any line in the \Mary Master Input File
beginning with an ! (explanation mark) is ignored by \Maryend. This feature
may be used to place comment lines (almost) anywhere in the Master Input
File. It may also be used to temporarily ``comment out'' various portions
of the Master Input File. This application is particularly useful for
making temporary changes in the \#labor component of the Master Input
File.\index{comment out}

There is one caution with the use of interspersed comments. They must not
be inserted between the name line for an element/command (in the \#menu
component) and the parameter line or parameter lines associated with that
name line. See section 5.7.2. Correspondingly, if a name line is to be
``commented out'', its associated parameter line or parameter lines must
also be ``commented out''.

\section{Beam Mode} The \#beam component is a required component for every
Master Input File. PREP automatically sets all the entries of the \#beam
component equal to zero at the beginning of a PREP run. (This is a
permissible set of values for certain kinds of \Mary runs.) The user can
modify the contents of the \#beam component by using PREP in {\em beam}
mode.

When in beam mode, PREP first asks for a Particle Type Code, as shown
below:

\begin{center}
\begin{tabular}{cl} Particle & \\ Type Code & Particle \\ \hline e &
electron or positron \\ p & proton or antiproton \\ h & H minus ion \\ d &
deuteron \\ mu & muon \\ pi & pion \\ o & other
\end{tabular}
\end{center} When type code {\em o} is used, the user is asked to specify
charge (in units of e) and mass (in MeV/$\mbox{c}^2$). The charge should be
input as a positive number. (\Mary works with the absolute value of the
input charge in any case.)

Next PREP will request that the user input the design momentum (in MeV/c)
or the design {\em kinetic} energy (in MeV) of the beam. Using these
values, PREP computes the magnetic rigidity and design relativistic beta
and gamma factors of the beamline. There is also provision for altering
these values.

Finally, PREP requests that the scale length be input (see section 4.1.2).
It then returns to {\em control} mode.

\subsection{Beam Editor}\index{beam} There is no editor for the \#beam component. The
proposed content of the Master Input File as prepared by PREP up to any
given point may be examined by entering the command {\em summ}. The \#beam
component can be changed at any time, if desired, by entering the command
{\em beam}, and then repeating the procedure of section 5.6 above.

If one wishes to bypass the use of PREP, as most experienced \Mary users do, the contents of the \#beam component can be set directly within a \Mary run by using a command with type code {\em cbm}.  See section 7.3.9.

\subsection{Sample Conversation in Beam Mode} A sample conversation with
PREP, specifying a proton beam with an energy of 800 MeV, and specifying a
scale length of 2 meters, is shown below:
\begin{footnotesize}
\begin{verbatim}
Typecmts,beam,menu,line,lump,loop,labor,summ,file,exit,abort or ?:
beam
beam input mode.
choose particle from listing:
     e (electron, positron)
     p (proton, anti-proton)
     h (H- ion)
     d (deuteron)
     mu (muon)
     pi (pion)
     o (other)
p

Absolute change (in units of e) is    1.000000000000000
Mass (in MeV/c*c) is   938.2723100000000
input design momentum in MeV/c
(note: if < 0, this will be assumed to be the design
         kinetic energy in MeV) :
-800

design energy and momentum:
kinetic energy (MeV) =     800.0000000000000
momentum (MeV/c) =     1463.296175078716
prep results:
values for brho,beta,gamma,(gamma-1), and abs(q/e) are
    4.881030646470486
   0.8418106683634142
    1.852630938240094
   0.8526309382400936
    1.000000000000000
type 1 if these are ok; type 0 to input your own values
1
lastly, input the scale length in meters :
2
Control mode.
\end{verbatim}
\end{footnotesize}
\subsection{Format of \#beam Component}\index{beam} If the {\em summ} command is issued
at this point, one obtains the result shown below for the proposed contents
of the Master Input File:
\begin{footnotesize}
\begin{verbatim}
Type cmts,beam,menu,line,lump,loop,labor,summ,file,exit,abort or ?:
summ
#comment
  This is a demonstration of the use of PREP.
#beam
   4.881030646470486
  0.8526309382400936
   1.000000000000000
   2.000000000000000
#menu
Control mode.
\end{verbatim}
\end{footnotesize} Observe that the \#beam component consists of 4
numbers:\index{beta} \index{design energy} \index{design momentum} \index{scale length} \index{mass} \index{charge}

\begin{tabbing}
\indent \= Second number \= - \= \kill \> First number \> - \> Brho, the
magnetic rigidity, in Tesla meters.\index{brho} \index{rigidity}\\ \> Second number \> - \> the quantity
(gamma-1) where gamma is the \\ \> \> \> relativistic gamma factor.\\ \>
Third number \> - \> the quantity $\mid$q/e$\mid$ , the absolute value of
the\\ \> \> \> charge in units of the electron charge.\\ \> Fourth number
\> - \> the scale length in meters.\index{scale length} \index{gamma}
\end{tabbing}
\noindent Since \Mary reads file 11 in free format, these numbers may be
put on separate lines if space for full precision is required, or may be
truncated and put on a single line if compactness is desired. PREP puts one
number on each line.

The reader may be curious why the quantity (gamma-1) is specified rather
than simply gamma. The answer has to do with numerical precision and round
off. \Mary computations require both the relativistic beta and the
relativistic gamma factors. Both these quantities can be computed to
machine precision from a knowledge of (gamma-1) even in extreme cases of
very small or very large particle energy.

Finally, note that the quantity Brho is related to the particle charge $q$ and the design momentum $p^0$ (see Section 4.1.2) by the equation
\begin{equation}
{\rm Brho} = p^0/|q|.
\end{equation}

\section{Menu Mode} Much of PREP is devoted to the specification of the
contents of a beam line or accelerator lattice, and the commands to be
executed. Elements of a lattice (as well as of lines, lumps, and loops) and
commands are drawn from a user specified {\em menu}.\index{menu}  A {\em menu} is a
table of elements, commands, and procedures. Each element, command, or
procedure is given a user specified name, a type code mnemonic, and various
required parameters.\index{mnemonic} \index{type code} The user specified name may be composed of one to
eight alpha-numeric characters. (It must, however, not begin with a number
or with the symbols , + $-$ \# ! * .) Subsequent references to each
element, command, or procedure are made by the user specified name. Thus,
the user may refer to elements, commands, or procedures in a way that
identifies their particular purpose.

User specified names must be unique. That is, {\em menu} entries, {\em
lines}, {\em lumps}, and {\em loops} must all have different names. (See
sections 5.8, 5.9, and 5.10 for discussion of {\em lines}, {\em lumps}, and
{\em loops}, respectively.)

Every Master Input File must have a \#menu component. It can be prepared by
using PREP in {\em menu} mode. As presently configured, the total number of
elements, commands, and procedures placed in the \#menu component should
not exceed 400.

When in the {\em menu} mode, PREP asks for the user specified name, then
the type code mnemonic, and finally the required parameters. A list of
currently available type codes is given below. Element type codes are
described in detail in section 6, and command type codes are described in
detail in sections 7 and 8. The type codes for procedures and fitting and
optimization commands are described in section 9. Note that all type codes
are given in lower case. If entries are made in upper case, they are
automatically converted to lower case by PREP and \Maryend.


\newpage
\noindent {\large{\bf Currently available type codes:}}
\vspace{5mm}
\begin{table}[h]

\noindent Mnemonics for elements (see Sect. 6):

\begin{center}
\begin{tabular}{lll} drft &: & drift\\ nbnd &: & normal entry and exit
bending magnet\\ pbnd &: & parallel faced bending magnet\\ gbnd &: &
general bending magnet\\ prot &: & rotation of reference plane\\ gbdy &: &
body of a general bending magnet\\ frng &: & hard edge dipole fringe
fields\\ cfbd &: & combined function bend\\ cfrn &: & combined function
dipole fringe fields\\ sol &: & solenoid\\ quad &: & quadrupole\\ cfqd &: &
combined function quadrupole\\ recm &: & REC quadrupole multiplet\\ sext &:
& sextupole\\ octm &: & mag. octupole\\ octe &: & elec. octupole\\ srfc &:
& short rf cavity\\ arot &: & axial rotation\\ thlm &: & thin lens low
order multipole\\ cplm &: & ``compressed'' low order multipole\\ twsm &: &
linear matrix via twiss parameters\\ dism &: & dispersion matrix\\ jmap &:
& j mapping\\ rmap &: & random map\\ mark &: & marker\\ dp &: & data
point\\ spce &: & space\\ arc &: & arc\\ usr1 \ldots usr10 &: & user
subroutines that act on phase space data\\ usr11 \ldots usr20 &: & user
subroutines that produce or act on maps\\ r**** &: & random counterpart of
element ****
\end{tabular}
\end{center}
\end{table}

\begin{table}

Mnemonics for simple commands (see Sect. 7):

\begin{center}
\begin{tabular}{lll} end &: & halt execution\\ of &: & open files\\ cf &: &
close files\\ rt &: & ray trace\\ num &: & number lines in a file\\ circ &:
& circulate\\ wcl &: & write contents of a loop\\ rapt &: & aperture
particle distribution with rectangular aperture\\ eapt &: & aperture
particle distribution with elliptic aperture\\ wnd &: & window a beam\\
wnda &: & window a beam in all planes\\ whst &: & write history of beam
loss\\ pmif &: & print contents of master input file\\ ptm &: & print
transfer map\\ tmi &: & input transfer map from an external file\\ tmo &: &
output transfer map to an external file\\ ps1 \ldots ps9 &: & parameter
sets\\ rps1 \ldots rps9 &: & random parameter sets\\ wps &: & write
parameter set\\ stm &: & store the existing transfer map\\ gtm &: & get
transfer map from storage\\ mask &: & mask off selected portions of
transfer map\\ ftm &: & filter transfer map\\ sqr &: & square the existing
map\\ symp &: & symplectify matrix in transfer map\\ iden &: & identity
mapping\\ inv &: & invert\\ rev &: & reverse map\\ revf &: & reverse
factorize\\ tran &: & transpose\\ tpol &: & twiss polynomial\\ dpol &: &
dispersion polynomial\\ time &: & write time\\ cdf &: & change output drop
file\\ bell &: & ring bell at terminal\\ wmrt &: & write out value of merit
function\\ paws &: & pause\\ inf &: & change or write out values of
infinities\\ zer &: & change or write out values of zeroes \\ cbm
&: & change or write out beam parameters\\ dims &: & dimensions\\ wuca &: &
write out contents of ucalc array\\ pli &: & path length information \\ shoa &: & show contents of arrays
\end{tabular} \end{center} \end{table}

\begin{table} Mnemonics for advanced commands (see Sect. 8):

\begin{center}
\begin{tabular}{lll} cod &: & compute off-momentum closed orbit data\\ tasm
&: & twiss analyze static map\\ tadm &: & twiss analyze dynamic map\\ ctr
&: & change tune range\\ snor &: & static normal form analysis\\ dnor &: &
dynamic normal form analysis\\ asni &: & apply script $N$ inverse\\ rasm &:
& resonance analyze static map\\ radm &: & resonance analyze dynamic map\\
sia &: & static invariant analysis\\ dia &: & dynamic invariant analysis\\
psnf &: & compute power of static normal form\\ pdnf &: & compute power of
dynamic normal form\\ pnlp &: & compute power of nonlinear part\\ fasm
&: & fourier analyze static map\\ fadm &: & fourier analyze dynamic map\\
pold &: & polar decompose matrix portion of transfer map\\ ppa &: &
principal planes analysis\\ tbas &: & translate basis\\ exp &: & compute
exponential\\ gbuf &: & get buffer contents\\ amap &: & apply map to a
function or moments\\ moma &: & moment analysis\\ bgen &: & generate beam\\
tic &: & translate initial conditions\\ smul &: & multiply polynomial by a
scalar\\ padd &: & add two polynomials\\ pmul &: & multiply two
polynomials\\ pb &: & Poisson bracket two polynomials\\ mn &: & compute
matrix norm\\ psp &: & polynomial scalar product\\ pval &: & evaluate a
polynomial\\ sq &: & select quantities\\ wsq &: & write selected
quantities\\ csym &: & check symplectic condition\\ geom &: & compute
geometry of a loop\\ fwa &: & copy file to a working array\\ lnf &: &
compute logarithm of normal form\\ merf &: & merge files \end{tabular}
\end{center} \end{table}

\begin{table}
\indent Mnemonics for procedures and fitting and optimization commands (see
Sect. 9):

\begin{center}
\begin{tabular}{lll} bip &: & begin inner procedure\\ bop &: & begin outer
procedure\\ tip &: & terminate inner procedure\\ top &: & terminate outer
procedure\\ aim &: & specify aims\\ vary &: & specify quantities to be
varied\\ fit &: & fit to achieve aims\\ mss &: & minimize sum of squares
optimization\\ opt &: & general optimization\\ mrt0 &: & merit function
(least squares)\\ mrt1 ... mrt5 &: & merit functions (user written)\\ con1
... con5 &: & constraints\\ grad &: & compute gradient matrix\\ scan &: &
scan parameter space\\ cps1 ... cps9 &: & capture parameter set\\ fps &: &
free parameter set\\ flag &: & change or write out values of flags and
defaults\\rset &: & reset menu entries
\end{tabular}
\end{center}
\end{table}
\clearpage

\subsection{Sample Conversation in Menu Mode} Below is a sample
conversation with PREP while in menu mode. The user has prepared a menu
consisting of elements and commands.

The elements are a long and short drift (user names {\em drl\/} and {\em
drs}\/), a horizontally focusing and a horizontally defocusing quadrupole
(user names {\em hfq} and {\em hdq}\/), and a parallel faced dipole (user
name {\em bend}\/).

The commands are to print the Master Input File (user name {\em
fileout}\/), to print the transfer map (user name {\em mapout}\/), to write
the transfer map on an external file (user name {\em wrtmap}\/), to replace
the transfer map by the identity map (user name {\em clear}\/), and an end
command (user name {\em bye}\/).

Also shown is the return to {\em control} mode by way of the {\em menu}
editor.

\begin{footnotesize}
\begin{verbatim}
Type cmts,beam,menu,line,lump,loop,labor,summ,file,exit,abort or ?:
menu
menu input mode
input name and mnemonic for menu item#   1
(or type# to go to menu edit mode)
drl drft
input length of drift in meters
2.28
 drl      drft      parameters:
  2.280000000000000
input name and mnemonic for menu item#   2
(or type # to go to menu edit mode)
drs drft
input length 1 of drift in meters
.45
 drs      drft      parameters:
 0.4500000000000000
input name and mnemonic for menu item #   3
(or type # to go to menu edit mode)
hfq quad
input length(m), field grad.(t/m), lfrn, tfrn
where: field grad .ge. 0 for hor focussing quad,
       field grad .lt. 0 for hor defocussing quad
       lfrn=1 for leading edge fr fields, =0 otherwise
       tfrn=1 for trailing edge fr fields,=0 otherwise
.5  2.7  1  1
 hfq     quad    parameters:
 0.5000000000000000     2.700000000000000     1.000000000000000
  1.000000000000000
input name and mnemonic for menu item #   4
(or type # to go to menu edit mode)
hdq quad
input length(m), field grad.(t/m), lfrn, tfrn
where: field grad .ge. 0 for hor focussing quad,
       field grad .lt. 0 for hor defocussing quad
       lfrn=1 for leading edge fr fields, =0 otherwise
       tfrn=1 for trailing edge fr fields,=0 otherwise
.5  -1.9 1 1
 hdq    quad    parameters:
 0.5000000000000000      -1.900000000000000      1.000000000000000
  1.000000000000000
input name and mnemonic for menu item #   5
(or type # to go to menu edit mode)
bend  pbnd
input bend angle(deg), gap size (m), normalized field integral, b(tesla)
36  0  .5  1.2
 bend    pbnd   parameters:
  36.00000000000000   0.0000000000000000E+00   0.5000000000000000
  1.200000000000000
input name and mnemonic for menu item #   6
(or type # to go to menu edit mode)
fileout pmif
input itype,ifile,isend where:
     itype=0 to print contents as they are
          =1 to print as interpreted by MARYLIE
          =2 to print (interpreted) #labor component
      ifile= external output file number
      isend=1 to print on file6 (tty)
           =2 to print on file ifile
           =3 to do both
1 12  3
 fileout    pmif    parameters:
  1.000000000000000    12.00000000000000   3.000000000000000
input name and mnemonic for menu item #   7
(or type # to go to menu edit mode)
mapout ptm
input nm,nf,nt,nu,ibasis for ptm
1 1 0 0 1
 mapout     ptm     parameters:
  1.000000000000000         1.000000000000000   0.0000000000000000E+00
 0.0000000000000000E+00     1.000000000000000
input name and mnemonic for menu item #   8
(or type # to go to menu edit mode)
trace rt
input icfile,nfcfile,norder,ntrace,nwrite,ibrief:
13  14  5  1000  5  0
 rays     rt     parameters:
  13.00000000000000     14.00000000000000     5.000000000000000
  1000.000000000000     5.000000000000000     0.0000000000000000E+00
input name and mnemonic for menu item #   9
(or type # to go to menu edit mode)
wrtmap tmo
there are  1 parameters for type tmo
input ifile, the file number on which map is to be written.
16
 wrtmap   tmo   parameters:
  16.00000000000000
input name and mnemonic for menu item #   10
(or type # to go to menu edit mode)
clear iden
there are  0 parameters for type iden
no parameters needed for reset to identity
 clear   iden   parameters:
 none
input name and mnemonic for menu item #   11
(or type # to go to menu edit mode)
bye end
there are 0 parameters for type end
no parameters for end command
 bye     end     parameters:
 none
input name and mnemonic for menu item #   12
(or type # to go to menu edit mode)
#
menu edit mode
type p, pl, i, ? or ctrl :
ctrl
Control mode.
\end{verbatim}
\end{footnotesize}


\subsection{Format of \#menu Component} If the {\em summ} command is issued
at this point one obtains the results shown below for the proposed contents
of the Master Input File:

\begin{footnotesize}
\begin{verbatim}
Type cmts,beam,menu,line,lump,loop,labor,summ,file,exit,abort or ?:
summ
#comment
 This is a demonstration of the use of PREP.
#beam
  4.881030646470486
 0.8526309382400936
  1.000000000000000
  2.000000000000000
#menu
 drl     drft
   2.28000000000000
 drs    drft
  0.450000000000000
 hfq    quad
  0.500000000000000       2.70000000000000          1.00000000000000
   1.00000000000000
  hdq   quad
   0.500000000000000     -1.90000000000000          1.00000000000000
    1.00000000000000
  bend    pbnd
   36.00000000000000     0.0000000000000000E+00     0.5000000000000000
   1.200000000000000
  fileout    pmif
    1.00000000000000     12.0000000000000          3.00000000000000
  mapout    ptm
    1.00000000000000        1.00000000000000     0.000000000000000E+00
   0.000000000000000E+00    1.00000000000000
  trace    rt
    13.0000000000000     14.0000000000000           5.00000000000000
    1000.00000000000     5.00000000000000          0.000000000000000E+00
  wrtmap    tmo
    16.0000000000000
  clear    iden
  bye     end
Control mode.
\end{verbatim}
\end{footnotesize}


From the contents of the \#menu component, one sees that each {\em menu}
item has the following format:
\begin{quotation} A ``name'' line giving the user specified name and type
code mnemonic for an element or command. Names may contain up to 8
characters.\index{name}

One or more associated ``parameter'' lines following the ``name'' line.
They specify the parameters required by the particular type code. Since
\Mary reads file 11 in free format, the parameters may occur on separate
lines, or may be grouped together. For convenience, PREP puts at most 3
parameters on a line. If a particular type code requires no parameters,
then there is no associated parameter line.
\end{quotation}
\subsection{Menu Editor} When in menu input mode, the contents of any menu
item may be changed simply be re-entering its name and then following
instructions. Thus, if a mistake is made while entering some menu item, one
should
\begin{itemize}
\item complete entering that particular menu item and its parameters,
\item then re-enter the name of that item and follow instructions.
\end{itemize}

There is also a simple {\em menu} editor to facilitate examination of the
\#menu component. It is activated by typing `\#'. The available commands
for the \#menu editor are listed below.
\begin{table}[hbp]
\begin{center}
\begin{tabular}{|c|l|l|} \hline
\multicolumn{3}{|c|}{Menu Editor Commands}\\ \hline Command & Usage &
Result\\ \hline p & p(CR) & {\underline p}rint the entire menu.\\ & &\\ pl
& pl(CR) & {\underline p}rint menu {\underline l}ines (entries) m through
n.\\ & m,n(CR) &\\ & &\\ i & i(CR) & Return to menu {\underline i}nput
mode.\\ & &\\ ? & ?(CR) & Get help from PREP.\\ & &\\ ctrl & ctrl(CR) &
Return to control mode.\\ \hline
\end{tabular}
\end{center}
\end{table}

Modifications in the \#menu component may be made at any time by returning
to the {\em menu} mode by way of the {\em control} mode. If extensive
modifications are required, it may be easier to make them by using a local
Editor on files 10 or 11 after completion of a PREP run.

\section{Line Mode} \index{lines} A {\em line} is a collection of elements and/or
commands referenced by a single user specified name. All possible \#menu
items, save for commands with type codes {\em bip}, {\em bop}, {\em tip},
and {\em top}, may appear in a line. See sections 9.1, 9.2, 9.3, and 9.4
Line names may contain up to 8 characters. (They must, however, not begin
wih a number or with the symbols , + $-$ \# ! * .) All names must be
unique. That is, {\em menu} entries, {\em lines}, {\em lumps}, and {\em
loops} must all have different names. (See sections 5.7, 5.9, and 5.10 for
a discussion of {\em menu} entries, {\em lumps}, and {\em loops},
respectively.)

The use of the lines feature of \Mary makes it possible to define
superelements and supercommands.

Lines may contain other lines up to a logical depth of 20. Lines may also
contain {\em lumps}. (See section 5.9 for a discussion of {\em lumps}.) The
contents of a line must meet the requirement that upon translation into a
string of elements and/or commands, all items in the string must appear in
the {\em menu} (or be {\em lumps}). (See section 5.7 for a description of
the \#menu component of the Master Input File.)

The user may define up to 100 lines, with up to 100 entries in each.
(Actually, the total number of lines, lumps, and loops must not exceed
100.)

The \#lines component is an optional component of the Master Input File.
For examples of the use of {\em lines}, see chapters 2 and 10.

\subsection{Sample Conversation in Line Mode} Below is a sample
conversation with PREP while in {\em line} mode. The user has defined three
lines with the user specified names {\em cell}, {\em ring}, and {\em
study}. The line {\em cell} contains a collection of elements drawn from
the menu. The line {\em ring} is defined in terms of the line {\em cell},
and contains 10 cells. The line {\em study} is a collection of elements and
commands. Note that PREP first asks for the name of a {\em line}, and then
for its contents. Return to {\em control} mode is made by way of the {\em
line} editor.

\begin{footnotesize}
\begin{verbatim}
Type cmts,beam,menu,line,lump,loop,labor,summ,file,exit,abort or ?:
line
line input mode
input name for the next line (or type # to edit)
cell
input contents of cell    (use & to cont on next line):
drl hdq drs bend drs hfq drl
input name for the next line (or type # to edit)
ring
input contents of ring     (use & to cont on next line):
10*cell
input name for the next line (or type # to edit)
study
input contents of study    (use & to cont on next line):
clear drl mapout hdq mapout clear
input name for the next line (or type # to edit)
#
line edit mode
type p, pl, i, ? or ctrl :
ctrl
Control mode.
\end{verbatim}
\end{footnotesize}


\subsection{Format of \#lines Component} If the {\em summ} command is
issued at this point, one gets the results listed below for the proposed
contents of the Master Input File.

\begin{footnotesize}
\begin{verbatim}
Type cmts,beam,menu,line,lump,loop,labor,summ,file,exit,abort or ?:
summ
#comment
 This is a demonstration of the use of PREP.
#beam
  4.881030646470486
 0.8526309382400936
  1.000000000000000
  2.000000000000000
#menu
 drl     drft
   2.28000000000000
 drs     drft
  0.450000000000000
 hfq    quad
  0.500000000000000     2.70000000000000     1.00000000000000
   1.00000000000000
 hdq    quad
  0.500000000000000    -1.90000000000000     1.00000000000000
   1.00000000000000
 bend    pbnd
  36.00000000000000    0.0000000000000000E+00  0.5000000000000000
   1.200000000000000
 fileout    pmif
   1.00000000000000     12.0000000000000     3.00000000000000
 mapout    ptm
   1.00000000000000     1.00000000000000     0.000000000000000E+00
  0.000000000000000E+00     1.00000000000000
 trace    rt
  13.0000000000000     14.0000000000000     5.00000000000000
  1000.00000000000     5.00000000000000      0.000000000000000E+00
 wrtmap    tmo
   16.0000000000000
 clear    iden
 bye    end
#lines
 cell
     1*drl     1*hdq     1*drs     1*bend     1*drs     &
     1*hfq     1*drl
 ring
     10*cell
 study
     1*clear     1*drl     1*mapout     1*hdq     1*mapout     &
     1*clear
#lumps
#loops
Control mode.
\end{verbatim}
\end{footnotesize} From the content of the \#lines component, one sees that
each line has the following format:
\begin{quotation} A ``name'' entry giving the user specified name of the
line. Names may contain up to 8 characters.

One or more associated ``contents'' entries on lines following the name
entry line. These entries specify the contents of the {\em line}. Each such
``contents'' entry line must end with an `\&' sign if it is to be followed
by another ``contents'' entry line. Since \Mary reads file 11 in free
format, the contents of a {\em line} may occur on separate entry lines, or
may be grouped together. For convenience, PREP puts at most 5 entries on a
line.
\end{quotation}

\subsection{Multiplicative Notation} \index{multiplicative notation} If a line contains several of the same
items in a row, this fact may be represented by the use of a multiplicative
(repetition number) notation. For example, if {\em hfq} is the user
specified name for a particular quadrupole, then two such quadrupoles in a
row may be specified by writing either hfq hfq or 2*hfq. By default, both
in PREP and \Mary $\!\!$, hfq means the same as 1*hfq.

Multiplicative notation may also be used with lines and lumps. For example
if {\em name} is the user specified name of a line or a lump, then m*name
(with m$>$ 0) has the same effect as m successive occurrences of name.

If name is the user specified name for an element, command, line, or lump
(within a line/lump), then 0*name has the same effect as omitting name.
This feature is useful for temporarily ``commenting out'' an item.

Negative repetition numbers may also be used. Their effect requires a more
extensive description. First, by default in both PREP and \Mary, $-$name
means the same thing as $-$1*name. Second, if name is an element, command,
or {\em previously constructed} lump, then m*name and $-$m*name have the
same effect. (See section 5.9 for a description of how and when lumps are
constructed.)

The case of lines and unmade lumps is somewhat more complicated. Suppose,
for example, that {\em name} is a line with entries a,b,c where a,b,c are
elements and/or commands. Then, as described above, 2*name is equivalent to
the sequence a b c a b c. However, $-$1*name is equivalent to the sequence
c b a, and $-$2*name is equivalent to the sequence c b a c b a. That is, a
negative repetition number for a line produces a {\em reversed} sequence.

The situation with unmade lumps is similar. Suppose, for example, that name
is a lump that has not been previously constructed, and that name has
entries a,b,c where a,b,c are elements and/or commands. Then 1*name
constructs and stores (and also concatenates) the map associated with the
sequence a b c. Similarly 2*name produces the same map, but concatenates it
twice. This, of course, is equivalent to concatenating the square of the
map. By contrast, $-$1*name constructs, stores, and concatenates the map
associated with the {\em reversed} sequence c b a.

The use of negative repetition numbers with lines/lumps within other
lines/lumps is still more complicated. Consider, for example, the case when
the lines {\em line1}, {\em line2}, and {\em line3} are defined by the
following \#line entries:

\begin{footnotesize}
\begin{verbatim}
     #lines
          line1
               a b c
          line2
               -1*line1
          line3
               -1*line2
\end{verbatim}
\end{footnotesize} Here a,b,c are elements and/or commands as before. Then,
as described above, {\em line2} is equivalent to the reversed sequence c b
a. Correspondingly, {\em line3} is equivalent to the ``reversed reversed''
sequence a b c. That is, lines/(unmade) lumps nested within other
lines/(unmade) lumps ``inherit'' negative signs from the repetition numbers
of the lines/(unmade) lumps they are nested within. The general rule for
lines and unmade lumps is that nested repetition numbers (including their
signs) behave as if they were simply multiplied together. One consequence
of this rule is that the actual map that is produced and stored when a lump
is constructed depends on the context in which the lump is found. In
general, negative repetition numbers should be used with care.

\subsection{Line Editor} When in {\em line} input mode, the name and
contents of any {\em line} item may be changed simply by re-entering its
name and then following instructions. Thus, if a mistake is made while
entering a {\em line} item, one should
\begin{itemize} \item complete entering the name of the line and at least
one item in its contents, \item then re-enter the name of that line and
follow instructions.
\end{itemize}

There is also a simple {\em line} editor to facilitate examination of the
\#lines component. It is activated by typing `\#'. The available commands
for the \#lines editor are listed below.
\begin{table}
\begin{center}
\begin{tabular}{|c|l|l|} \hline
\multicolumn{3}{|c|}{Line Editor Commands}\\ \hline Command & Usage &
Result\\ \hline p & p(CR) & {\underline p}rint the entire contents of\\ & &
the \#lines component.\\ & &\\ pl & pl(CR) & {\underline p}rint {\underline
l}ines m through n.\\ & m,n(CR) &\\ & &\\ i & i(CR) & Return to menu
{\underline i}nput mode.\\ & &\\ ? & ?(CR) & Get help from PREP.\\ & &\\
ctrl & ctrl(CR) & Return to {\em control} mode.\\ \hline
\end{tabular}
\end{center}
\end{table}

Modifications in the \#lines component may be made at any time by returning
to the {\em line} mode by way of the {\em control} mode. If extensive
modifications are required, it may be easier to make them by using a local
Editor on files 10 or 11 after completion of a PREP run.

\section{Lump Mode}\index{lumps} A {\em lump} is a collection of elements and/or
commands whose ultimate result is the construction by \Mary of a map. This
map is stored by
\Maryend, and is available for later use. Each lump is given a user
specified name, and each time the name of a particular lump is invoked, the
corresponding map is recalled from memory.

Lump names may contain up to 8 characters. (They must, however, not begin
with a number or with the symbols , + $-$ \# ! * .) All names must be
unique. That is, {\em menu} entries, {\em lines}, {\em lumps}, and {\em
loops} must all have different names. (See sections 5.7, 5.8, and 5.10 for
a discussion of {\em menu} entries, {\em lines}, and {\em loops},
respectively.)

Lumps may contain lines and other lumps up to a logical depth of 20. (See
section 5.8 for a description of the \#lines component of the Master Input
File.) A lump must meet the requirement that upon translation into a string
of elements and/or commands, all items in the string must appear in the
{\em menu} or be other lumps. (See section 5.7 for a description of the
\#menu component of the Master Input File.)

\Mary constructs the map associated with a lump as follows:
\begin{itemize} \item A storage area is set aside for the map to be
associated with the lump, and the identity map is initially stored in this
area. \item The contents of the lump are translated into a string of
elements and/or commands. Each time an element is encountered in the
string, the map for that element is computed and concatenated with whatever
map is in the storage area. The map in the storage area is then replaced by
the result of this concatenation. (See section 3.11.2 for a description of
the concatenation of maps.) Whenever another lump is encountered in the
string, the map for that lump is concatenated with whatever map is in the
storage area. The map in the storage area is then replaced by the result of
this concatenation. Whenever a command is encountered in the string, it is
examined to see whether or not it is of the type that acts on a map. If it
is not, the command is ignored, and a message to this effect is written out
by \Mary at the terminal. (A ray trace command is an example of a command
that does not act on a map.) If the command is of the type that acts on a
map, the command is applied to the map in the storage area. The map in the
storage area is then replaced by the map resulting from the application of
this command. [A square command (type code mnemonic {\em sqr}) is an
example of a command that acts on a map.]
\end{itemize} Thus after all the items in the string associated with the
contents of the lump have been encountered, the result is a stored map.

The user may define up to 100 lumps with up to 100 entries in each.
(Actually, the total number of lines, lumps, and loops must not exceed
100.) In actual operation, \Mary does not construct and store the map for a
lump unless and until it is required as a result of some entry in the
\#labor component of the Master Input File. See section 5.11. As presently
configured, at any given moment the total allowed number of stored maps
associated with lumps is 20. If the user tries to construct more than 20
lumps, some of the existing (already constructed) lumps are destroyed on a
first made first destroyed basis. The assumption is that recently
constructed lumps are more likely to be required for subsequent
calculations than ancient lumps. Section 5.9.3 describes how the user may
control the destruction of lumps in a completely flexible way if required.

The \#lumps component is an optional component of the Master Input File.
For examples of the use of {\em lumps}, see sections 2.7 and 10.4.

\subsection{Sample Conversation in Lump Mode} Below is a sample
conversation with PREP while in {\em lump} mode. The user has defined two
lumps called {\em lmpbend} and {\em lmpcell}. The lump {\em lmpbend}
consists of the map for the element {\em bend} listed in \#menu. The lump
{\em lmpcell} consists of the map corresponding to all the elements in the
line {\em cell} when their individual maps are concatenated together. The
line {\em cell} is assumed to be defined in the \#lines component of the
Master Input File. Note that PREP first asks for the name of a {\em lump},
and then for its contents. Return to {\em control} mode is made by the way
of the {\em lump} editor.

\begin{footnotesize}
\begin{verbatim}
Type cmts,beam,menu,line,lump,loop,labor,summ,file,exit,abort or ?:
lump
lump input mode
input name for the next lump (or type # to edit)
lmpbend
input contents of lmpbend (use & to cont on next line):
bend
input name for the next lump (or type # to edit)
lmpcell
input contents of lmpcell (use & to cont on next line):
cell
input name for the next lump (or type # to edit)
#
lump edit mode
type p, pl, i, ? or ctrl :
ctrl
Control Mode.
\end{verbatim}
\end{footnotesize}

\subsection{Format of the \#lumps Component} If the {\em summ} command is
issued at this point, one gets the results listed below for the proposed
contents of the Master Input File.

\begin{footnotesize}
\begin{verbatim}
Type cmts,beam,menu,line,lump,loop,labor,summ,file,exit,abort or ?:
summ
#comment
 This is a demonstration of the use of PREP.
#beam
  4.881030646470486
 0.8526309382400936
  1.000000000000000
  2.000000000000000
#menu
 drl    drft
   2.28000000000000
 drs drft
  0.450000000000000
 hfq   quad
  0.500000000000000     2.70000000000000     1.00000000000000
   1.00000000000000
 hdq   quad
  0.500000000000000    -1.90000000000000     1.00000000000000
   1.00000000000000
 bend    pbn
  36.00000000000000     0.0000000000000000E+00     0.5000000000000000
  1.200000000000000
 fileout    pmif
   1.00000000000000     12.0000000000000     3.00000000000000
 mapout    ptm
   1.00000000000000    1.00000000000000    0.000000000000000E+00
  0.000000000000000E+00     1.00000000000000
 trace    rt
   13.0000000000000     14.0000000000000     5.00000000000000
   1000.00000000000     5.00000000000000      0.000000000000000E+00
 wrtmap tmo
   16.0000000000000
 clear   iden
 bye end
#lines
 cell
     1*drl     1*hdq     1*drs     1*bend     1*drs     &
     1*hfq     1*drl
 ring
    10*cell
 study
     1*clear     1*drl     1*mapout     1*hdq     1*mapout     &
     1*clear
#lumps
 lmpbend
     1*bend
 lmpcell
     1*cell
#loops
Control mode.
\end{verbatim}
\end{footnotesize} From the content of the \#lumps component one sees that
each lump has the following format:
\begin{quotation} A ``name'' line giving the user specified name of the
{\em lump}. Names may contain up to 8 characters.

One or more associated ``contents'' lines following the name line. These
lines specify the contents of the {\em lump}. Each such ``contents'' line
must end with an `\&' sign if it is to be followed by another ``contents''
line. Since \Mary reads file 11 in free format, the contents of a {\em
lump} may occur on separate entry lines, or may be grouped together. For
convenience, PREP puts at most 5 entries on a line.
\end{quotation}

\subsection{Multiplicative Notation}\index{multiplicative notation} If a lump appears repeatedly, then it
may either be written repeatedly, or a repetition number may be used. For
example, if {\em name} is a lump, then 2*name is equivalent to name name.
By default, both in PREP and in \Mary, name means the same as 1*name.

The case of 0*name, when {\em name} is a lump, is more complicated. If
0*name appears within a line, lump, or a loop, then it is ignored. The same
is true if 0*name appears in \#labor (see section 5.11) and {\em name} has
{\em not} actually been constructed. (See section 5.9 for a description of
how and when lumps are constructed.) This feature is useful for temporarily
``commenting out'' a lump.

By contrast, if {\em name} has been constructed previously and 0*name
appears in \#labor, then the map for this lump is destroyed at that point
in the \Mary run, and its associated storage allocation is made available
for the construction and storage of the map for another lump. This \Mary
feature places the construction and destruction of lumps (actually their
associated maps) under the direct control of the user.

Negative repetition may also be used (with care) for lumps. See section
5.8.3.

\subsection{Lump Editor} When in {\em lump} input mode, the name and
contents of any {\em lump} item may be changed simply by re-entering its
name and then following instructions. Thus, if a mistake is made while
entering a {\em lump} item, one should
\begin{itemize} \item complete entering the name of the lump and at least
one item in its contents, \item then re-enter the name of that lump and
follow instructions.
\end{itemize}

There is also a simple {\em lump} editor to facilitate examination of the
\#lumps component. It is activated by typing `\#'. The available commands
for the \#lumps editor are listed below.
\begin{table}
\begin{center}
\begin{tabular}{|c|l|l|}\hline
\multicolumn{3}{|c|}{Lump Editor Commands}\\ \hline Command & Usage &
Result\\ \hline p & p(CR) & {\underline p}rint the entire contents of\\ & &
the \#lumps component.\\ & &\\ pl & pl(CR) & {\underline p}rint {\underline
l}ines m through n.\\ & m,n(CR) &\\ & &\\ i & i(CR) & Return to lumps
{\underline i}nput mode.\\ & &\\ ? & ?(CR) & Get help from PREP.\\ & &\\
ctrl & ctrl(CR) & Return to {\em control} mode.\\ \hline
\end{tabular}
\end{center}
\end{table}

Modifications in the \#lumps component may be made at any time by returning
to the {\em lump} mode by way of the {\em control} mode. If extensive
modifications are required, it may be easier to make them by using a local
Editor on files 10 or 11 after completion of a PREP run.
\vspace{1in}
\pagebreak

\section{Loop Mode}\index{loops} A {\em loop} is a collection of elements, lines, and/or
lumps to be used for tracking or other studies. (See section 5.8 for a
description of {\em lines}, and section 5.9 for a description of {\em
lumps}.) The loop feature of \Mary permits tracking using more than one
transfer map. See sections 2.7, 10.3, 10.4, and 10.8. The loop feature also
is used in the computation of the geometry of a beamline or circular
machine. See sections 10.9 and 10.11 through 10.14.

Each loop is given a user specified name. Names may contain up to 8
characters. (They must, however, not begin with a number or with the
symbols , + $-$ \# ! * .) All names must be unique. That is, {\em menu}
entries, {\em lines}, {\em lumps}, and {\em loops} must all have different
names. (See sections 5.7, 5.8, and 5.9 for a discussion of {\em menu}
entries, {\em lines}, and {\em lumps}, respectively.)

A loop must meet the requirement that upon translation into a string of
elements and/or commands and/or lumps, all items in the string must either
appear in the {\em menu} or be lumps. (See section 5.7 for a description of
\#menu.) Although commands may appear either implicitly or explicitly in a
loop, they are, in fact, all ignored except for those that act on
phase-space data and those that occur within the definitions of lumps.
Loops may contain other loops up to a logical depth of 20.

As presently configured, the user may define up to 100 loops with up to 100
entries in each. (Actually, the total number of lines, lumps, and loops
must not exceed 100.) When a loop is completely translated into a string as
described above, the total number of resulting items must be less than
1000.

The \#loops component is an optional component of the Master Input File.
Loops are used in conjunction with commands having the type code mnemonics
{\em circ}, {\em wcl}, and {\em geom}. See sections 7.3, 7.33, and 8.38.
For examples of the use of {\em loops}, see Chapters 2 and 10.

\subsection{Sample Conversation in Loop Mode} Below is a sample
conversation with PREP while in {\em loop} mode. The user has defined two
loops with the names {\em turtle} and {\em hare}. The loop {\em turtle }
contains the line {\em cell}, and the loop {\em hare} contains the lump
{\em lmpcell}. The line {\em cell} is assumed to be defined in the \#lines
component of the Master Input File, and the lump {\em lmpcell} is assumed
to be defined in the \#lumps component of the Master Input File. Note that
PREP first asks for the name of a {\em loop}, and then for its contents.
Return to {\em control} mode is made by way of the {\em loop} editor.

\begin{footnotesize}
\begin{verbatim}
Type cmts,beam,menu,line,lump,loop,labor,summ,file,exit,abort or ?:
loop
loop input mode
input name for the next loop (or type # to edit)
turtle
input contents of turtle   (use & to cont on next line):
cell
input name for the next loop (or type # to edit)
hare
input contents of hare     (use & to cont on next line):
lmpcell
input name for the next loop (or type # to edit)
#
loop edit mode
type p, pl, i, ? or ctrl :
ctrl
control mode.
\end{verbatim}
\end{footnotesize}

\subsection{Format of the \#loops Component} If the {\em summ} command is
issued at this point, one gets the results listed below for the proposed
contents of the Master Input File.

\begin{footnotesize}
\begin{verbatim}
Type cmts,beam,menu,line,lump,loop,labor,summ,file,exit,abort or ?:
summ
#comment
 This is a demonstration of the use of PREP.
#beam
  4.881030646470486
 0.8526309382400936
  1.000000000000000
  2.000000000000000
#menu
 drl    drft
   2.28000000000000
 drs    drft
  0.450000000000000
 hfq    quad
  0.500000000000000     2.70000000000000     1.00000000000000
   1.00000000000000
 hdq    quad
  0.500000000000000     -1.90000000000000     1.00000000000000
   1.00000000000000
 bend    pbnd
   36.00000000000000     0.0000000000000000E+00     0.5000000000000000
   1.200000000000000
 fileout    pmif
   1.00000000000000     12.0000000000000    3.00000000000000
 mapout    ptm
   1.00000000000000     1.00000000000000    0.000000000000000E+00
  0.000000000000000E+00     1.00000000000000
 trace    rt
   13.0000000000000     14.0000000000000     5.00000000000000
   1000.00000000000    5.00000000000000     0.000000000000000E+00
 wrtmap tmo
   16.0000000000000
 clear    iden
 bye      end
#lines
 cell
     1*drl     1*hdq     1*drs     1*bend     1*drs     &
     1*hfq    1*drl
 ring
    10*cell
 study
     1*clear     1*drl    1*mapout     1*hdq     1*mapout     &
     1*clear
#lumps
 lmpbend
     1*bend
 lmpcell
     1*cell
#loops
 turtle
     1*cell
  hare
     1*lmpcell
Control mode.
\end{verbatim}
\end{footnotesize} From the content of the \#loops component, one sees that
each loop has the following format:
\begin{quotation} A ``name'' line giving the user specified name of the
{\em loop}. Names may contain up to 8 characters.

One or more associated ``contents'' lines following the name line. These
lines specify the contents of the {\em loop}. Each such ``contents'' line
must end with an `\&' sign if it is to be followed by another ``contents''
line. Since \Mary reads file 11 in free format, the contents of a {\em
loop} may occur on separate entry lines, or may be grouped together. For
convenience, PREP puts at most 5 entries on a line.
\end{quotation}

\subsection{Multiplicative Notation}\index{multiplicative notation} There is no reason for a loop to
appear repeatedly in the \#labor component of the Master Input File. See
section 5.11. Consequently, if {\em name} is a loop and m is a positive
integer, then m*name has the same effect as 1*name when appearing in
\#labor. The only exception is that an entry of the form 0*name is ignored.
Finally, by default both in PREP and \Mary, name means the same as 1*name.

Negative repetition numbers may be used with loops with care. By default,
$-$name has the same effect as $-$1*name.

If {\em name} is a loop and $-$name (or $-$1*name) appears in \#labor, the
effect is that {\em name} is constructed in reverse order. Also, consistent
with the description above, $-$m*name has the same effect as $-$1*name.
Note that if a loop has a negative repetition number, then lines, {\em
unmade} lumps, and loops inside this loop will inherit the negative sign
associated with the negative repetition number. See section 5.8.3.

It is possible to have repeated loops inside other loops. In this case,
$-$m*name gives the same results as taking the contents of m*name in
reversed order (providing that whatever lumps are referenced have not been
constructed previously). Specifically, repetition numbers for loops within
other loops behave just as they do for nested lines. See section 5.8.3.

\subsection{Loop Editor} When in {\em loop} input mode, the name and
contents of any {\em loop} item may be changed simply by re-entering its
name and then following instructions. Thus, if a mistake is made while
entering a {\em loop} item, one should
\begin{itemize} \item complete entering the name of the {\em loop} and at
least one item in its contents, \item then re-enter the name of that {\em
loop} and follow instructions.
\end{itemize}

There is also a simple {\em loop} editor to facilitate examination of the
\#loops component. It is activated by typing `\#'. The available commands
for the \#loops editor are listed below.

\begin{table}[hb]
\begin{center}
\begin{tabular}{|c|l|l|} \hline
\multicolumn{3}{|c|}{Loop Editor Commands}\\ \hline Command & Usage &
Result\\ \hline p & p(CR) & {\underline p}rint the entire contents of\\ & &
the \#loops component.\\ & &\\ pl & pl(CR) & {\underline p}rint {\underline
l}ines m through n.\\ & m,n(CR) &\\ & &\\ i & i(CR) & Return to loops
{\underline i}nput mode.\\ & &\\ ? & ?(CR) & Get help from PREP.\\ & &\\
ctrl & ctrl(CR) & Return to {\em control} mode.\\ \hline
\end{tabular}
\end{center}
\end{table}

Modifications in the \#loops component may be made at any time by returning
to the {\em loop} mode by way of the {\em control} mode. If extensive
modifications are required, it may be easier to make them by using a local
Editor on files 10 or 11 after completion of a PREP run.

\section{Labor Mode}\index{labor} As indicated earlier, the \#labor component specifies
the actual system to be studied and what operations are actually to be
carried out. Thus, the \#labor component is a list of elements, commands,
procedures, lines, lumps, and loops. As presently configured, this list may
contain up to 1000 entries.

The entries in the \#labor component are drawn from the \#menu, \#lines,
\#lumps, and \#loops components. That is, all entries in the \#labor
component must also occur elsewhere as user specified names in at least one
of the \#menu, \#lines, \#lumps, and \#loops components. PREP checks to
make sure that this is the case. {\em Moreover, the user specified names of
all entries in the \#menu, \#lines, \#lumps, and \#loops components must be
unique}.

During program execution, \Mary reads the entries in the labor list
sequentially. It is the entries of the \#labor component that determine
what will actually be done in a given \Mary run. For this reason, the
\#labor component is a required component for every Master Input File.

Throughout a run, \Mary keeps in memory a transfer map that will be
referred to as the {\em total transfer map}.\index{transfer map} \index{total transfer map} The map consists of matrix and
Lie generator parts plus path length. At the beginning of a run, the total
transfer map is set equal to the identity map. When \Mary encounters the
name of an {\em element} as it reads the labor list, it calculates the
transfer map for that element (based on the parameters specified for that
element in \#menu). Specifically, it computes the numerical values for the
entries of the matrix $R$ and the polynomials $f_3$ and $f_4$ that
represent the transfer map. It also computes path length and stores the
result in $f(0)$. Next, it concatenates this map with the {\em total
transfer map}, and stores the result as the {\em new total transfer map}.
Path length is also accumulated. (See section 3.11.2 for a description of
the concatenation of maps.) Finally, \Mary discards both the transfer map
for the element it has just encountered and the previous total transfer
map, and retains only the new total transfer map. Thus,
\begin{quote}
\bf MaryLie concatenates as it steps through the labor list, maintains the
new (current) total transfer map in storage, and discards all previous
maps.
\end{quote} Following concatenation, the next entry will be read from the
labor list.\index{concatenation}

Similarly, when \Mary reads the name of a lump in the labor list, it
computes (or recalls from memory if previously computed) the transfer map
for that lump. The transfer map for the lump is then concatenated with the
total transfer map to give a new transfer map.

When \Mary reads an entry that is a {\em command}, it will do whatever
operation is requested (e.g., a ray trace or a map manipulation). It then
again reads the next entry from the labor list.

If \Mary encounters the name of a line in the labor list, it carries out
all commands and concatenates all successive maps associated with this
line.

This whole process stops when \Mary encounters an entry in the labor list
with the type code {\em end}. (See section 7.1.) Thus, each labor list must
have somewhere an entry with the type code {\em end}. Usually this is the
last entry in the list. However, it may occur earlier. In that case, all
subsequent entries in the labor list are ignored.

At the beginning of this section, it was pointed out that it is the entries
of the \#labor component that determine what will actually be done in a
given \Mary run. This feature of \Mary makes it possible for the
\#menu, \#lines, \#lumps, and \#loops components to have entries that are
{\em not} used in a given \Mary run. Indeed, \Mary is organized in such a
way that maps for elements in \#menu are not computed unless they are
referenced either explicitly or implicitly in \#labor. Similarly, lumps are
not constructed, lines are not translated, etc., unless they are actually
needed. Thus, the user may make a variety of \Mary runs simply by changing
the contents of the \#labor component.

For examples of various \Mary runs, and how they are specified by the
contents of \#labor, see chapters 2 and 10.

\subsection{Sample Conversation in Labor Mode} Below is a sample
conversation with PREP while in {\em labor} mode. The user has specified a
\Mary run that computes a variety of maps, some of which are written on an
external file for later use, and then repeatedly traces rays through a
simple cell.

Note that PREP asks for the entries in the labor list one at a time. Return
to {\em control} mode is made by way of the {\em labor} editor.

\begin{footnotesize}
\begin{verbatim}
Type cmts,beam,menu,line,lump,loop,labor,summ,file,exit,abort or ?:
labor
labor input mode:
input next labor entry (type '#' to edit)
fileout
input next labor entry (type '#' to edit)
study
input next labor entry (type '#' to edit)
ring
input next labor entry (type '#' to edit)
wrtmap
input next labor entry (type '#' to edit)
clear
input next labor entry (type '#' to edit)
lmpcell
input next labor entry (type '#' to edit)
wrtmap
input next labor entry (type '#' to edit)
trace
input next labor entry (type '#' to edit)
bye
input next labor entry (type '#' to edit)
#
labor edit mode
type p,pc,pl,u,n,d,dl,i,ib,?, or ctrl :
ctrl
Control mode.
\end{verbatim}
\end{footnotesize}

\subsection{Format of the \#labor Component} If the {\em summ} command is
issued at this point, one gets the results listed below for the proposed
contents of the Master Input File.

\begin{footnotesize}
\begin{verbatim}
Type cmts,beam,menu,line,lump,loop,labor,summ,file,exit,abort or ?:
summ
#comment
 This is a demonstration of the use of PREP.
#beam
  4.881030646470486
 0.8526309382400936
  1.000000000000000
  2.000000000000000
#menu
 drl    drft
   2.28000000000000
 drs    drft
  0.450000000000000
 hfq    quad
  0.500000000000000      2.70000000000000      1.00000000000000
   1.00000000000000
 hdq    quad
  0.500000000000000     -1.90000000000000     1.00000000000000
   1.00000000000000
 bend    pbnd
  36.00000000000000     0.0000000000000000E+00     0.5000000000000000
  1.200000000000000
 fileout    pmif
   1.00000000000000      12.0000000000000      3.00000000000000
 mapout    ptm
   1.00000000000000      1.00000000000000      0.000000000000000E+00
  0.000000000000000E+00      1.00000000000000
 trace    rt
   13.0000000000000      14.0000000000000      5.00000000000000
   1000.00000000000      5.00000000000000      0.000000000000000E+00
 wrtmap    tmo
   16.0000000000000
 clear   iden
 bye     end
#lines
 cell
     1*drl     1*hdq     1*drs     1*bend     1*drs     &
     1*hfq      1*drl
 ring
    10*cell
 study
     1*clear     1*drl     1*mapout     1*hdq     1*mapout     &
     1*clear
#lumps
 lmpbend
      1*bend
 lmpcell
      1*cell
#loops
 turtle
     1*cell
 hare
     1*lmpcell
#labor
    1*fileout
    1*study
    1*ring
    1*wrtmap
    1*clear
    1*lmpcell
    1*wrtmap
    1*trace
    1*bye
Control mode.
\end{verbatim}
\end{footnotesize}

From the contents of the \#labor component, one sees that it consists of a
simple list with {\em one} entry on each line. Indeed, \Mary reads the
\#labor component one line at a time, and expects only one entry on each
line. Thus, unlike other components of the Master Input File, the {\em
\#labor component must have only one entry on each line}. Since \Mary reads
each line in free format, the entry may begin anywhere on the line.

\subsection{Multiplicative Notation}\index{multiplicative notation} As described previously, menu entries,
lines, lumps, and (by mistake) loops may appear repeatedly in \#labor. This
can be achieved more compactly by writing m*name where m is an integer and
{\em name} is a menu entry, line, lump, or loop. See sections 5.8.3, 5.9.3,
and 5.10.3 for more detail. Finally, name means the same as 1*name.

\subsection{Labor Editor} PREP has a simple line editor for correcting
mistakes in the \#labor component. To assist the user, the editor has a
``viewing window'' that sits over what is referred to as the ``current''
line. The editor for \#labor is activated by typing `\#'. The available
commands for the \#labor editor are listed below.

\begin{table}
\begin{center}
\begin{tabular}{|c|l|l|} \hline
\multicolumn{3}{|c|}{Labor Editor Commands} \\ \hline
 Command &   Usage     & Result \\ \hline
    p    &   p(CR)     & {\underline p}rint all of \#labor component with line \\
         &             & numbers.  After this command is executed \\
         &             & the current line is the last line. \\
         &             & \\
   pc    &   pc(CR)    & {\underline p}rint {\underline c}urrent line. \\
         &             & \\
   pl    &   pl(CR)    & {\underline p}rint {\underline l}ines m through n. After this \\
         &   m,n(CR)   & command is executed, the current line  \\
         &             & is line n.  Thus, this command may be  \\
         &             & used with m=n to place the viewing \\
         &             & window over line n. \\
         &             & \\
    u    &   u(CR)     & Move viewing window {\underline u}p one line. \\
         &             & \\
    n    &   n(CR)     & Move viewing window to the {\underline n}ext line. \\
         &             & \\
    d    &   d(CR)     & {\underline d}elete the current line.  Viewing  \\
         &             & window now sits over the next line.  \\
         &             & \\
   dl    &   dl(CR)    & {\underline d}elete {\underline l}ines m through n.  Viewing \\
         &   m,n(CR)   & window now sits over the old line n+1. \\
         &             & \\
    i    &   i(CR)     & {\underline i}nput after the current line.  PREP  \\
         &             & goes into {\em labor} input mode.  To \\
         &             & return to the {\em labor} editor, type \#. \\
         &             & \\
   ib    &   ib(CR)    & {\underline i}nput {\underline b}efore line m. PREP goes into\\
         &   m(CR)     & {\em labor} input mode.  To return to the \\
         &             & {\em labor} editor, type \#. \\
         &             & \\
    ?    &   ?(CR)     & Get help from PREP. \\
         &             & \\
  ctrl   &   ctrl(CR)  & Return to {\em control} mode. \\ \hline
\end{tabular}
\end{center}
\end{table}

     To leave the {\em labor} editor and return to {\em control} mode,
type {\em ctrl}.
Modifications in the \#labor component may be made at any time by returning
to the {\em labor} mode by way of the {\em control} mode.

     If extensive changes in the \#labor component are required, it may be
easier to make them by using a local Editor on files 10 or 11 after
completion of a PREP run.

\section{Use of the Summary Command in PREP}
     While in {\em control} mode, the user may type {\em summ} to obtain a listing of
what has been accomplished to date in a PREP session.  Illustrations of the
use of the {\em summ} command are given in sections 5.6.3, 5.7.2, 5.8.2, 5.9.2,
5.10.2, and 5.11.2.  In essence, the {\em summ} command gives a listing of what
PREP has prepared to date to put on file 10.

\section{Use of the File Command in PREP}
     While in {\em control} mode, the user may type the command {\em file} at any time
during a PREP run.  This command causes file 10 to be rewound, and then
causes PREP to write on file 10 whatever has been specified to date about
the \#comment, \#beam, \#menu, \#lines, \#lumps, \#loops, and \#labor components
of the Master Input File.  When a PREP run is terminated by means of the
{\em exit} command, the final results of the PREP run are automatically written
on file 10.  If a PREP run is terminated with the {\em abort} command, this final
writing on file 10 is bypassed.

     It is recommended that the {\em file} command be entered from time to time
in the course of a PREP run.  Then, should the computer crash, all that
might be lost are the entries put in after the last {\em file} command.

     An example of the use of the {\em file} command is shown below.

\begin{footnotesize}
\begin{verbatim}
Type cmts,beam,menu,line,lump,loop,labor,summ,file,exit,abort or ?:
file
master input file written on file 10
Control mode.
\end{verbatim}
\end{footnotesize}

\section{Terminating and Resuming a PREP Run}
     A PREP run may be terminated by entering the commands {\em exit} or {\em abort}
while in {\em control} mode.  Use of the {\em exit} command is shown below.

\begin{footnotesize}
\begin{verbatim}
Type cmts,beam,menu,line,lump,loop,labor,summ,file,exit,abort or ?:
exit
master input file written on file 10
End of PREP.
\end{verbatim}
\end{footnotesize}
As explained in section 5.13, use of the {\em exit} command automatically results
in the issuing of a final {\em file} command before termination.  Use of the
{\em abort} command causes termination without the issue of a final {\em file} command.

     A partially satisfactory or partially complete file 10 or 11 can be
used as input at the beginning of a PREP run.  In this way, PREP can be
used to edit or complete a file 10.  The user simply copies the existing
file 10 or 11 into file 9, and then reads file 9 at the beginning of a PREP
run.  See section 5.3.  Shown below is a sample PREP conversation in which
use has been made of file 9.

\begin{footnotesize}
\begin{verbatim}
Do you want instructions? (y,n) <n>
n
Read a PREP data set from file 9? (y,n,?) <n>
y
file 9 read in.
Control mode.
\end{verbatim}
\end{footnotesize}

     If the {\em summ} command is issued at this point, one obtains the output
shown below thus indicating that PREP has been informed of the current
state of the Master Input File, and is ready to receive corrections and/or
additions.

\begin{footnotesize}
\begin{verbatim}
Type cmts,beam,menu,line,lump,loop,labor,summ,file,exit,abort or ?:
summ
#comment
 This is an example of the use of PREP.
#beam
  4.881030646470486
 0.8526309382400936
  1.000000000000000
  2.000000000000000
#menu
 drl      drft
   2.28000000000000
 drs      drft
  0.450000000000000
 hfq      quad
  0.500000000000000       2.70000000000000       1.00000000000000
   1.00000000000000
 hdq      quad
  0.500000000000000      -1.90000000000000       1.00000000000000
   1.00000000000000
 bend     pbnd
  36.00000000000000      0.0000000000000000E+00  0.5000000000000000
  1.200000000000000
 fileout  pmif
   1.00000000000000       12.0000000000000       3.00000000000000
 mapout   ptm
   1.00000000000000       1.00000000000000      0.000000000000000E+00
  0.000000000000000E+00   1.00000000000000
 trace    rt
   13.0000000000000       14.0000000000000       5.00000000000000
   1000.00000000000       5.00000000000000      0.000000000000000E+00
 wrtmap   tmo
   16.0000000000000
 clear    iden
 bye      end
#lines
 cell
     1*drl         1*hdq         1*drs         1*bend        1*drs      &
     1*hfq         1*drl
 ring
    10*cell
 study
     1*clear       1*drl         1*mapout      1*hdq         1*mapout   &
     1*clear
#lumps
 lmpbend
     1*bend
 lmpcell
     1*cell
#loops
 turtle
     1*cell
 hare
     1*lmpcell
#labor
    1*fileout
    1*study
    1*ring
    1*wrtmap
    1*clear
    1*lmpcell
    1*wrtmap
    1*trace
    1*bye
Control mode.
\end{verbatim}
\end{footnotesize}

