%%!TEX root = ./marylie.tex
%%--.----1----.----2----.----3----.----4----.----5----.----6----.----7----.-!

%\blankpage
\chapter{Introduction to MaryLie}

\section{Brief Program Description}
\Mary 3.0 is the first version of the MARYland LIE algebraic particle beam
transport and tracking program.  It is named after Henrietta Maria, wife of England's King Charles I, and Sophus Lie.  Lord Baltimore called the English colony that was to become the state of Maryland ``Terra Maria''.  Sophus Lie was the discoverer of Lie algebras and Lie groups.  The purpose of \Mary is to aid in the design
and evaluation of both linear charged particle beam transport systems and
circulating storage rings.  The program employs algorithms based on a Lie
algebraic formulation of charged particle trajectory calculations, and is able to
compute transfer maps for and trace rays through single or multiple beam-line
elements.  This is done for the full 6-dimensional phase space {\em without} the
use of numerical integration or traditional matrix methods.  All nonlinearities,
including chromatic effects, through third (octupole) order are included.  Thus
\Mary 3.0 includes effects one order higher than those usually handled by existing
matrix-based programs.  In addition, \Mary is exactly symplectic (canonical)
through all orders.  The version of \Mary described in this manual consists of approximately 40,000 lines of code organized into approximately 500 routines.

\section{Program Performance and Accuracy}
\Mary may be used both for particle tracking around or through a lattice and for
analysis of linear and nonlinear lattice properties.  It may also be used
to transport moments without the need for particle tracking.  When used for tracking, it
is both versatile and extremely fast.  Tracking can be performed element by
element, lump by lump, or any mixture of the two.  (A lump is a collection of
elements combined together and treated by a single transfer map.)  The speed for
element by element tracking is comparable to that of other tracking codes.  When
collections of elements can be lumped together to form single transfer maps,
tracking speeds can be orders of magnitude faster.

Major portions of \Mary 3.0 have been benchmarked against numerical integration.
All were verified to be exact through third order.  When properly used \Mary has
an absolute accuracy of four or more significant figures.  (In many cases,
depending on the amplitude of  betatron oscillations, the absolute accuracy is six
or more significant figures.)  That is, \Mary has an accuracy comparable to or
better than the accuracy with which accelerator and beamline components are
currently measured.

\Mary also has extremely powerful analytic tools.  They include the calculation of
tunes and first and second-order chromaticities and anharmonicities  (dependence of tunes on betatron
amplitudes) and first, second, and third-order temporal dispersion
(dependence of closed-orbit transit time on energy, sometimes called the
phase-slip factor, and related to momentum compaction, the dependence of
the closed-orbit  length on energy);  first, second, and third-order
ordinary dispersion, and all other linear lattice functions and their energy dependence
through second order; nonlinear lattice functions; nonlinear phase-space distortion; transfer map normal forms;
nonlinear resonance driving terms; nonlinear invariants; expected Fourier
coefficients for tracking data; and moment data including eigenemittances
and high-order moments.

To facilitate actual machine design, \Mary has extensive fitting,  scanning, and
optimization routines.  These routines make it possible to  fit, scan, or optimize
essentially all quantities that can be computed by \Mary including those computed
in user written subroutines.  \Mary also provides full geometrical
information including the orientation and location of beamline elements
in 3-dimensional space.

Finally, \Mary can be used to give an explicit representation for the linear and
nonlinear properties of the total transfer map of a system.  This information can
be used to evaluate or improve the optical quality of a single pass system such as
a beam transport line or linear collider.  For a circulating system such as a
storage ring, it is hoped that the explicit knowledge in this case of the one-turn
transfer map can eventually be used to predict the dynamic aperture without the
need for extensive long-term tracking.

\section{Future Versions of MaryLie}
Because \Mary 3.0 is the first version of a Lie algebraic code, it does not have
all the features one might possibly desire.  For example, \Mary 3.0 does not treat
beam element placement, alignment, and powering errors.

     Future versions of \Mary will have features not found in this first version.
Possible improvements include the treatment of additional beam-line elements;
handling of magnet placement, alignment, and powering errors; acceleration;
space-charge effects; etc.  These future versions will become available later as
\Mary 3.1, \Mary 3.2, etc.

     One of the advantages of the Lie algebraic method is that it is readily
extended, in principle, to higher orders.  It is expected that it will eventually
be possible to go to fifth order.  This will be the subject of \Mary 5.0 and its
later versions, 5.1, 5.2, etc.  Even higher order calculations appear possible,
but will require considerably more work.  There is no insurmountable difficulty in treating idealized beam-line elements in higher order.  The major problem is the realistic treatment of fringe-field and magnet error effects, etc., which is essential for high-order expansions to make sense.


\section{Auxiliary Programs}
\Mary has been written to make use of several external files both for input and
output.  This feature makes it possible to design and use other computer programs
both to prepare input for \Mary and to postprocess \Mary output.  Currently
several such programs exist or are in preparation.  A brief list of these programs
is given below.  Users may also wish to write additional programs of their own.

%\numberbysubsection
\subsection{Programs for Preparation of MaryLie Input}
\begin{itemize}
\item  PREP\@.  This program may be used to prepare the Master Input File
         for \Maryend.  Its operation is described in later chapters of this
         manual.\index{PREP}
\item   MADMARY and MARYMAD\@.  MADMARY takes a lattice, as specified in MAD
         format, and translates it into the current \Mary format.
         MARYMAD does the reverse.  \index{MAD}
\item   GENMAP\@.\index{GENMAP}  In normal operation, \Mary uses an internal library of
         idealized beamline elements.  However, \Mary also has the
         capability to read in and use externally specified transfer maps.
         GENMAP refers to a series of programs that take as input real
         (calculated or measured) magnetic and/or electric field data for
         some beam-line element, and then generate by numerical
         integration the transfer map describing motion through
         that element.  This transfer map can
         subsequently be used by \Maryend.  In this way, it is possible to
         use \Mary to treat real beamline elements including the effect
         of fringe fields, magnetic imperfections, etc.  \Mary also has a
		 built-in GENMAP routines that are used for various purposes.  See
		 sections 6.8, 6.23, 6.27, and 8.7.  Finally, the user can in principle
		 write routines of his/her own that call the internal GENMAP routines.
		 See section 6.21
\end{itemize}

\subsection{Programs for Processing of MaryLie Output}
\begin{itemize}
\item  GENERAL PLOTTING\@.  When performing ray traces, \Mary reads phase-space
         initial conditions from an external file and writes the final
         phase-space coordinates on another external file.  \Mary also
		 writes various other kinds of data to external files.  All these
		 quantities can be plotted against each other to make phase-space or
		 other kinds of plots by use of any of the commonly available plotting
		 programs that read data files in columnar form.  See sections 2.2
		 through 2.7, 10.3, 10.4, 10.7, 10.8, 10.9, 10.11, and 10.12.

\item  POSTER is a general purpose Post-Script plotting program written
by C. Thomas Mottershead.\index{POSTER}  As such, it can be used to produce the general
kinds of plots described above.  In addition, POSTER can read files
produced by the \Mary {\em geom} command.  See section 8.38.  With this
capacity POSTER can be used to produce composite plots that include a
schematic representation of the underlying beamline, and to produce
beamline layout drawings.  See sections 10.13 and 10.14.
\end{itemize}

%\numberbysection
\section{Program Availability}
Both \Mary and PREP are written in {\sc Fortran 77}.  Serial scalar versions
are currently available for any computer that supports {\sc Fortran 77}.  In addition vectorized and/or parallel versions of \Mary (and
serial versions of PREP) are available for certain vectorized/parallel
machines.

All of \Mary 3.0 is believed to conform to Fortran 77 standards except for the use of {\em include} statements (which are supported by most Fortran 77 computers) and the timing routine {\em cputime}.  The latter is invoked by the \Mary command {\em time} (see section 7.29).  It is machine dependent, and must be set up by the user if it is to be employed.



