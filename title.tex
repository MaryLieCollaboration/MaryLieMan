%%!TEX root = ./marylie.tex

\begin{titlepage}
\begin{center}
{\Huge MARYLIE 3.0}

\vspace{.15in}
{\huge Users' Manual}

\vspace{10mm}
{\LARGE  A Program for Charged Particle Beam Transport}

\vspace{.15in}
{\LARGE  Based on Lie Algebraic Methods
\footnote{Work supported in part by U.S. Department of Energy
Grant DEFG02-96ER40949.}}

\vspace{20mm}


\begin{tabular}{cc}
\large Alex J. Dragt                  & \large Robert D. Ryne \\
\vspace{-2mm}\\
Center for Theoretical Physics        & Lawrence Berkeley National
Laboratory \\
University of Maryland                & Berkeley, California  94720 \\
College Park, Maryland  20742

\vspace{5mm}\\
                                      & \large Filippo Neri\\
\large David R. Douglas               & \large C. Thomas Mottershead \\
\vspace{-2mm}\\
Thomas Jefferson National             & Los Alamos National Laboratory \\
Accelerator Facility                  & Los Alamos, New Mexico \ 87545 \\
Newport News, Virginia \ 23606        \\

\vspace{5mm}\\
\large \'Etienne Forest               & \large Liam M. Healy\\
\vspace{-2mm}\\
KEK Laboratory                        & Naval Research Laboratory\\
Tsukuba, Ibaraki, Japan \ 305         & Washington, DC \ 20375\\
\vspace{5mm}\\

{\large Petra Sch\"{u}tt}             & {\large Dan T. Abell}\\
\vspace{-2mm}\\
Technische Universit\"{a}t Darmstadt  & RadiaSoft LLC\\
D-64289 Darmstadt, Germany            & Boulder, Colorado \ 80303\\

\end{tabular}

\vspace{10mm}
{\large October 2021}
\end{center}
\end{titlepage}

{\small
\centerline{\bf Disclaimer}
MaryLie 3.0 consists of a Main Program and approximately 500 Subroutines.
Together they comprise approximately 40,000 lines of what is intended,
wherever practical, to be standard FORTRAN 77 code.  As time has
permitted, considerable care has been expended in checking many of these
subroutines, and some are at least partially self checking.  (For
example, the product of two one-meter drift maps should be a two-meter
drift map.)  Other MaryLie features and their associated subroutines are
still under development.

Wherever possible, the user should also make both independent and
internal consistency checks of his/her own.  Programs should not be run
blindly.  There is no substitute for understanding.  The authors would
appreciate receiving information concerning any detected errors or
difficulties.  A form for this purpose is provided in Chapter 12.

In any case, the MaryLie 3.0 program is provided ``as is'' without any
warranty of any kind.  No warranties, expressed or implied, are made
that the program and its procedures are free of error, or are consistent
with any particular standard of merchantability or fitnesss for a
particular purpose, or that they will meet
your requirements for any particular application.  They should not be
relied on for solving a problem whose incorrect solution could result in
injury to a person or loss of property.  If you use the program in such
a manner, it is at your own risk.  The authors and their agencies
disclaim all liability for direct, incidental, or consequential damage
resulting from your use of the program.

\vspace{.2in}
\centerline{\bf Acknowledgement}
Initial work on the use of Lie methods in the general area of Nonlinear
Dynamics began when the first author (AJD) was a sabbatical visitor in
spring 1973 at the Institut des Hautes Etudes Scientifiques (IHES) in the Paris suburb
Bures-sur-Yvette.  He is indebted to Louis Michel and the IHES for this
opportunity and their fine hospitality.  Some years later, during a
sabbatical visit to the Accelerator Technology Division of the Los Alamos National
Laboratory, he realized that Lie methods could be applied to the
computation and analysis of Charged Particle Beam Transport.  He is is
indebted to
Richard Cooper and all the other accelerator physicists at Los Alamos for their
hospitality then, and for their continued support and encouragement over
the ensuing years.  He is also grateful to the University of Maryland for
its continued support.  Finally, research on Lie algebraic methods has been
supported by the United States Department of Energy since 1980.  All the authors are
grateful to DOE and the American people for their nurture of fundamental and applied
science.

\vspace{.2in}
\centerline{\bf Assistance}
Several have made major contributions to the preparation of this manual
and its exhibits including Rachel Needle, Glenn Vanderwoude, and Maura and
Patrick Roberts.

\vspace{.2in}
\centerline{\bf Revision Record}
June 1998, February 1999, April 1999, October 1999, March 2000, May 2000, October 2002, May 2003, December 2003.

\vspace{.2in}
\centerline{\bf Availability}
See section 1.5.

\vspace{.35in}

\copyright 2003 Alex J. Dragt.  All rights reserved.}

